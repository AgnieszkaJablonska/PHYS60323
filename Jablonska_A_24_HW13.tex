\documentclass[12pt]{article}
\usepackage{epsfig}

\usepackage[
    starfontserif % comment for sans glyphs
    ]{starfont}
\DeclareSymbolFont{starfontsym}{OT1}{sts}{m}{n}
\DeclareMathSymbol{\mathSun}{\mathord}{starfontsym}{115}

\usepackage{times}
\usepackage{fancyhdr}
\usepackage{pslatex}
\usepackage{amsmath}
\usepackage{mathrsfs}
\usepackage[dvipsnames]{xcolor}
\usepackage[hidelinks]{hyperref}%renewcommand{\topfraction}{1.0}
\renewcommand{\topfraction}{1.0}
\renewcommand{\bottomfraction}{1.0}
\renewcommand{\textfraction}{0.0}
\setlength {\textwidth}{6.6in}
\hoffset=-1.0in
\oddsidemargin=1.00in
\marginparsep=0.0in
\marginparwidth=0.0in                                                                               
\setlength {\textheight}{9.0in}
\voffset=-1.00in
\topmargin=1.0in
\headheight=0.0in
\headsep=0.00in
\footskip=0.50in                                         
\setcounter{page}{1}
\begin{document}
\def\pos{\medskip\quad}
\def\subpos{\smallskip \qquad}
\newfont{\nice}{cmr12 scaled 1250}
\newfont{\name}{cmr12 scaled 1080}
\newfont{\swell}{cmbx12 scaled 800}


\begin{center}
{\LARGE
PHYS  20323/60323: Fall 2024 - LaTeX Example
}\\

%%%%%%%%%%%%%%%%%%%%%%%%%%%%%%%%%%%%%%%%%%%%%%%%%%%%%%%%%%%%
\end{center}
%%%%%%%%%%%%%%%%%%%%%%%%%%%%%%%%%%%%%%%%%%%%%%%%%%%%%%%%%%%%
% Section Heading
%%%%%%%%%%%%%%%%%%%%%%%%%%%%%%%%%%%%%%%%%%%%%%%%%%%%%%%%%%%%
\begin{enumerate}
    \item An electron is found to be in the spin state (in the \textit{z}-basis): $\chi$ = A $\begin{pmatrix} 3i \\ 4 \\ \end{pmatrix}$ 
\begin{enumerate}
    \item (5 points) Determine the possible values of A such that the state is normalized.
    \item (5 points) Find the expectation values of the operators \color{red} {$S_{x}$}\color{black}, \color{purple} {$S_{y}$}\color{black}, \color{orange} {$S_{z}$} \color{black} and $\vec{S^2}$.
\end{enumerate}
The matrix representations in the \textit{z}-basis for the components of electron spin operators are \\given by:\\ 


\color{red} {$S_{x}$} = $\frac{\hbar}{2}$ $\begin{pmatrix} 0 \phantom{kk} 1 \\ 1 \phantom{kk} 0 \\ \end{pmatrix}$ }; \phantom{kkk}     \quad \color{purple} {$S_{x}$} = $\frac{\hbar}{2}$ $\begin{pmatrix} 0  -i \\ i \phantom{kk} 0 \\ \end{pmatrix}$ }; \phantom{kkk}           \quad \color{orange} {$S_{x}$} = $\frac{\hbar}{2}$ $\begin{pmatrix} 1 \phantom{kk} 0 \\ 0 \; -1 \\ \end{pmatrix}$ };\\


\color{black}
\item The average electrostatic field in the earth's atmosphere in fair weather is approximately given:
\begin{align}
    \vec{E} = E_0 (Ae^{-\alpha z} + Be^{-\beta z})\hat{z,}
\end{align}
 

where A, B, $\alpha$, $\beta$ are positive constants and z is the highest above the (locally flat) earth surface. 
\begin{enumerate}
    \item (5 points) Find the average charge density in the atmosphere as a function of height\\
    \item (5 points) Find the electric potential as a function height above the earth.\\
\end{enumerate}    

\item\textbf{The following questions refer to stars the Table below.}\\
Note: There may be multiple answers.

\begin{tabular}{|c|c|c|c|c|c|}
\hline
     Name & Mass & Luminosity & Lifetime & Temperature & Radius \\
\hline
     $\beta$ Cyg. & $1.3 M_\mathSun$& $3.5 L_\mathSun$& & &\\
     \hline
     $\alpha$ Cen. & $1.0 M_\mathSun$ & & & & $1R_\mathSun$\\
     \hline
     $\eta$ Car. & $60. M_\mathSun$ & $10^6 L_\mathSun$ & $8.0\times 10^5$ years & &\\
     \hline
     $\sigma$ Eri. & $6.0 M_\mathSun$ & $10^3 L_\mathSun$& & $20,000K$ &\\
     \hline
     $\delta$ Scu. & $2.0 M_\mathSun$& & $5.0\times10^8$ years & & $2 R_\mathSun$\\
     \hline
     $\gamma$ Del. & $0.7 M_\mathSun$& &$4.5\times10^{10}$ years&$5000K$&\\
     \hline
\end{tabular}

\begin{enumerate}
    \item (4 points) Which of these stars will produce a planetary nebula.\\
    \item (4 points) Elements heavier then \textit{Carbon} will be produced in which stars.
\end{enumerate}    

\end{enumerate} 








\end{document}
